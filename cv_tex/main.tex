%%%%%%%%%%%%%%%%%%%%%%%%%%%%%%%%%%%%%%%%%
% Medium Length Professional CV
% LaTeX Template
% Version 2.0 (8/5/13)
%
% This template has been downloaded from:
% http://www.LaTeXTemplates.com
%
% Original author:
% Rishi Shah 
%
% Important note:
% This template requires the resume.cls file to be in the same directory as the
% .tex file. The resume.cls file provides the resume style used for structuring the
% document.
%
%%%%%%%%%%%%%%%%%%%%%%%%%%%%%%%%%%%%%%%%%

%----------------------------------------------------------------------------------------
%	PACKAGES AND OTHER DOCUMENT CONFIGURATIONS
%----------------------------------------------------------------------------------------

\documentclass{resume} % Use the custom resume.cls style

\usepackage[utf8]{inputenc}
\usepackage[left=0.75in,top=0.6in,right=0.75in,bottom=0.6in]{geometry} % Document margins
\usepackage[usenames,dvipsnames]{xcolor}
\usepackage[colorlinks = true, linkcolor = MidnightBlue, urlcolor  = MidnightBlue, citecolor = MidnightBlue, anchorcolor = MidnightBlue]{hyperref}
\usepackage{enumitem}
%----------------------------------------------------------------------------------------
%	HEADER SECTION
%----------------------------------------------------------------------------------------

\name{Thiago Holleben} % Your name
\address{\href{mailto:hollebenthiago@dal.ca}{hollebenthiago@dal.ca}~$\cdot$~ \href{https://www.mathstat.dal.ca/~thiagoh/}{website}}

\begin{document}
%----------------------------------------------------------------------------------------
%	EDUCATION SECTION
%----------------------------------------------------------------------------------------

\begin{rSection}{Education}
{\bf Universidade Federal do Rio de Janeiro, Brazil} \hfill {\em march 2016 - july 2017 (interrupted)} 
\\ Major in Mathematics\\
{\bf Universidade Federal do Rio de Janeiro, Brazil} \hfill {\em july 2017 - 2020} 
\\ Bachelor in Applied Mathematics
\\
\\{\bf Universidade Federal do Rio de Janeiro, Brazil} \hfill {\em august 2019 - december 2021} 
\\ Master's degree in Mathematics
\\ Dissertation title: \href{https://hollebenthiago.github.io/msc/main.pdf}{The log-concavity of chromatic polynomials}.
\\ Advisor: Seyed Hamid Hassanzadeh
\\
\\{\bf Dalhousie University, Canada} \hfill {\em september 2022 - current} 
\\ PhD in Mathematics
\\ Advisor: Sara Faridi

\end{rSection}

%----------------------------------------------------------------------------------------
%	TECHNICAL SSKILLS SECTION
%----------------------------------------------------------------------------------------


%--------------------------------------------------------------------------------
%    Projects SECTION
%-----------------------------------------------------------------------------------------------
% \begin{rSection}{Computational Algebra personal projects}
% {\bf Some simulations on statistical properties of Betti tables over a given class of ideals}
% \\
% Macaulay2 and Python script to study some statistical properties of Betti tables over a given class of ideals.
% \\
%  {\scriptsize Technologies used: Macaulay2, Python3, github repository:  \href{https://github.com/hollebenthiago/Quantitative-Betti-Tables}{Quantitative-Betti-Tables}}
% \\
% {\bf Algorithms related to Elliptic Curves, prime factorization and cryptography.}
% \\
% Webpage that includes some statistical properties of rational points of elliptic curves over finite fields related to Serre's bound, a simple cryptography scheme and Lenstra's algorithm on factorization.
% \\
%  {\scriptsize Technologies used: Javascript, link:  \href{https://hollebenthiago.github.io/ecc/}{Webpage}}

% \end{rSection}

%----------------------------------------------------------------------------------------
%	ACADEMIC SECTION
%----------------------------------------------------------------------------------------

\begin{rSection}{Academic Achievements}
{\bf Professor Michael Edelstein Memorial Graduate Prize (2023)}
\\
Annual prize awarded to a graduate student who shows great promise in the mathematical sciences. 


{\bf Honours undergraduate research project (2018)}
\\
Received an award for best undergraduate research project on the algebraic and combinatorial properties of Edge Ideals.
\\
Advisor: Seyed Hamid Hassanzadeh
\end{rSection}

\begin{rSection}{Papers}

\begin{enumerate}[label={}]
    \item Thiago Holleben. (2023). The weak Lefschetz property and mixed multiplicities of monomial ideals.  \href{https://arxiv.org/abs/2306.13274}{arxiv link}
    \item Susan M. Cooper, Sara Faridi, Lisa Nicklasson, Adam Van Tuyl, Thiago Holleben. (2023). The weak Lefschetz property of whiskered graphs. \href{https://arxiv.org/abs/2306.04393}{arxiv link}
    \item Sara Faridi, Thiago Holleben. Homological invariants of ternary graphs. (preprint, to be submitted October 2023).
    \item Thiago Holleben. Rees algebras and Lefschetz properties of squarefree monomial ideals. (in progress).
    \item T. Chau, A. Duval, S. Faridi, T. Holleben, S. Morey, L. \c{S}ega.  Powers of a simplex. (in progress).
\end{enumerate}
\end{rSection}
%----------------------------------------------------------------------------------------
%	POSITION OF RESPONSIBILITY SECTION
%----------------------------------------------------------------------------------------

\begin{rSection}{Presentations}
    \begin{enumerate}[label={}]
        \item Rees algebras and Lefschetz properties of squarefree monomial ideals. Stockholm Commutative Algebra seminar. 26th October 2023 (online talk)
        \item Lefschetz properties of squarefree monomial ideals. 10th HLF. September 2023 (poster, lightning talk)
        \item Lefschetz properties of squarefree monomial ideals. BrianFest. August 2023 (poster)
        \item Homological Invariants of ternary graphs. 34o CBM. July 2023 (poster)
        \item The weak Lefschetz property and mixed multiplicities of monomial ideals. MSRI/SLMath Summer School in Notre Dame. May 2023 (poster)
        \item Degree of the inverse of a birational monomial map. MSRI/SLMath Summer School, Notre Dame. May 2023 (lightning talk)
        \item The weak Lefschetz property and mixed multiplicities of monomial ideals. Lefschetz workshop, Fields institute. May 2023 (talk)
        \item Lefschetz properties and mixed multiplicities via hyperplane arrangements. ICTP workshop. May 2023 (poster)
        \item The weak Lefschetz property and mixed multiplicities of squarefree monomial ideals. ICTP summer school. May 2023 (talk)
        \item Homological invariants of ternary graphs. BIRS workshop. April 2023 (talk)
        \item Homological invariants of ternary graphs. Southern Regional Algebra Conference. 2023 (online talk)
        \item Homological invariants of ternary graphs. CAAC 2023. January 2023 (poster)
        \item The log-concavity of chromatic polynomials. UFRJ. July 2022 (talk, in portuguese)
        \item Exploring examples with Macaulay2. UFRJ. July 2022 (talk, in portuguese)
        \item Hilbert functions in Combinatorics. UFRJ. July 2022 (talk, in portuguese)
        \item A combinatorial interpretation of the primary decomposition of edge ideals. UFRJ undergraduate student seminar. 2019 (talk, in portuguese)
        \item A brief dictionary between Algebra and Combinatorics. SIAC-UFRJ 2018 (talk, in portuguese)
    \end{enumerate}
\end{rSection}

\begin{rSection}{Leadership positions}
{\bf Organizer of Week of Applied Mathematics and Mathematical Engineering (SEM$^2$Ap)}
\\
Organized virtual Week of Applied Mathematics and Mathematical Engineering (SEM$^2$Ap) in 2020
{\scriptsize More info on:  \href{http://semap.rio.br/2020/en/}{Webpage},   \href{https://www.youtube.com/channel/UC14NMQ5cOsSuLrAQGGa2T4Q}{Youtube}, 
\href{https://www.instagram.com/semap.rio/}{Instagram}}
\\
{\bf Teaching assistant at Dalhousie}
\\
Fall of 2022 and Winter of 2023
\\
{\bf Teaching assistant at UFRJ} 
\\
First semester of 2019
\\
{\bf Teaching assistant at UFRJ} 
\\
First and second semester of 2018
\\
{\bf Organizer of weekly student seminar at UFRJ}
\\
Organized weekly seminars at UFRJ during the second semester of 2019.
\end{rSection}

% \clearpage

%----------------------------------------------------------------------------------------
%	EVENTS ATTENDED
%----------------------------------------------------------------------------------------

\begin{rSection}{Events attended}

    \begin{itemize}[label={}]
        \item 10th Heidelberg Laureate Forum 2023 (Heidelberg University)
        \item BrianFest (University of Nebraska-Lincoln)
        \item 34o Colóquio Brasileiro de Matemática (IMPA)
        \item Commutative Algebra and its Interaction with Algebraic Geometry (University of Notre Dame) 2023
        \item Workshop on Lefschetz Properties in Algebra, Geometry, Topology and Combinatorics. (Fields Institute) 2023
        \item Workshop on Commutative Algebra and Algebraic Geometry in Prime Characteristics. (ICTP) 2023
        \item School on Commutative Algebra and Algebraic Geometry in Prime Characteristics. (ICTP) 2023
        \item Interactions Between Topological Combinatorics and Combinatorial Commutative Algebra (BIRS) 2023
        \item Combinatorial Algebra meets Algebraic Combinatorics 2023. (Waterloo University) 2023
        \item Combinatorial Algebra meets Algebraic Combinatorics 2020. (Dalhousie University) 2020
        \item 1st Joint Meeting Brazil-France in Mathematics. (IMPA) 2019
        \item 32o Colóquio Brasileiro de Matemática. (IMPA) 2019
        \item SIMCARA. (USP São Carlos) 2019
        \item Syzygies, from Theory to Applications. (UFPB) 2019
        \item XXV Escola de Álgebra. (UNICAMP) 2018
    \end{itemize}
% \\ 34o Colóquio Brasileiro de Matemática (IMPA) 2023 (poster)
% \\ Commutative Algebra and its Interaction with Algebraic Geometry (University of Notre Dame) 2023 (poster, lightning talk)
% \\ Workshop on Lefschetz Properties in Algebra, Geometry, Topology and Combinatorics. (Fields Institute) 2023 (talk, problem)
% \\ Workshop on Commutative Algebra and Algebraic Geometry in Prime Characteristics. (ICTP) 2023 (poster)
% \\ School on Commutative Algebra and Algebraic Geometry in Prime Characteristics. (ICTP) 2023 (talk)
% \\ Interactions Between Topological Combinatorics and Combinatorial Commutative Algebra (BIRS) 2023 (talk)
% % \\ Southern Regional Algebra Conference 2023. (online talk)
% \\ Combinatorial Algebra meets Algebraic Combinatorics 2023. (Waterloo University) 2023 (poster)%(\href{http://semap.rio.br}{poster} presented)
% \\ Combinatorial Algebra meets Algebraic Combinatorics 2020. (Dalhousie University) 2020
% \\ 1st Joint Meeting Brazil-France in Mathematics. (IMPA) 2019
% \\ 32o Colóquio Brasileiro de Matemática. (IMPA) 2019
% \\ SIMCARA. (USP São Carlos) 2019
% \\ Syzygies, from Theory to Applications. (UFPB) 2019
% \\ XXV Escola de Álgebra. (UNICAMP) 2018
% \\
\end{rSection}

\begin{rSection}{Skills}

    \begin{tabular}{ @{} >{\bfseries}l @{\hspace{6ex}} l }
    Programming:\ & Macaulay2, Sage, Python, Javascript, Julia
    \\
    \end{tabular}
    
    \end{rSection}

% \begin{rSection}{Research line}
% I am interested in the study of Combinatorial Commutative Algebra, Computational Algebra and their applications.

% \end{rSection}


\end{document}
