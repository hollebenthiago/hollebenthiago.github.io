%%%%%%%%%%%%%%%%%%%%%%%%%%%%%%%%%%%%%%%%%
% Medium Length Professional CV
% LaTeX Template
% Version 2.0 (8/5/13)
%
% This template has been downloaded from:
% http://www.LaTeXTemplates.com
%
% Original author:
% Rishi Shah 
%
% Important note:
% This template requires the resume.cls file to be in the same directory as the
% .tex file. The resume.cls file provides the resume style used for structuring the
% document.
%
%%%%%%%%%%%%%%%%%%%%%%%%%%%%%%%%%%%%%%%%%

%----------------------------------------------------------------------------------------
%	PACKAGES AND OTHER DOCUMENT CONFIGURATIONS
%----------------------------------------------------------------------------------------

\documentclass[12pt]{resume} % Use the custom resume.cls style

\usepackage[utf8]{inputenc}
\usepackage[left=0.75in,top=0.75in,right=0.75in,bottom=0.6in]{geometry} % Document margins
\usepackage[usenames,dvipsnames]{xcolor}
\usepackage[colorlinks = true, linkcolor = MidnightBlue, urlcolor  = MidnightBlue, citecolor = MidnightBlue, anchorcolor = MidnightBlue]{hyperref}
\usepackage{enumitem}
%----------------------------------------------------------------------------------------
%	HEADER SECTION
%----------------------------------------------------------------------------------------

\name{Thiago Holleben} % Your name
\address{\href{mailto:hollebenthiago@dal.ca}{hollebenthiago@dal.ca}~$\cdot$~ \href{https://www.mathstat.dal.ca/~thiagoh/}{website}}

\begin{document}
%----------------------------------------------------------------------------------------
%	EDUCATION SECTION
%----------------------------------------------------------------------------------------

\begin{rSection}{Education}
{\bf Universidade Federal do Rio de Janeiro, Brazil} \hfill {\em March 2016 - July 2017 (interrupted)} 
\\ Major in Mathematics\\
{\bf Universidade Federal do Rio de Janeiro, Brazil} \hfill {\em July 2017 - 2020} 
\\ Bachelor in Applied Mathematics
\\
\\{\bf Universidade Federal do Rio de Janeiro, Brazil} \hfill {\em August 2019 - December 2021} 
\\ Master's degree in Mathematics
\\ Dissertation title: \href{https://hollebenthiago.github.io/msc/main.pdf}{The log-concavity of chromatic polynomials}.
\\ Advisor: Seyed Hamid Hassanzadeh
\\
\\{\bf Dalhousie University, Canada} \hfill {\em September 2022 - current} 
\\ PhD in Mathematics
\\ Advisor: Sara Faridi

\end{rSection}

%----------------------------------------------------------------------------------------
%	TECHNICAL SSKILLS SECTION
%----------------------------------------------------------------------------------------


%--------------------------------------------------------------------------------
%    Projects SECTION
%-----------------------------------------------------------------------------------------------
% \begin{rSection}{Computational Algebra personal projects}
% {\bf Some simulations on statistical properties of Betti tables over a given class of ideals}
% \\
% Macaulay2 and Python script to study some statistical properties of Betti tables over a given class of ideals.
% \\
%  {\scriptsize Technologies used: Macaulay2, Python3, github repository:  \href{https://github.com/hollebenthiago/Quantitative-Betti-Tables}{Quantitative-Betti-Tables}}
% \\
% {\bf Algorithms related to Elliptic Curves, prime factorization and cryptography.}
% \\
% Webpage that includes some statistical properties of rational points of elliptic curves over finite fields related to Serre's bound, a simple cryptography scheme and Lenstra's algorithm on factorization.
% \\
%  {\scriptsize Technologies used: Javascript, link:  \href{https://hollebenthiago.github.io/ecc/}{Webpage}}

% \end{rSection}

%----------------------------------------------------------------------------------------
%	ACADEMIC SECTION
%----------------------------------------------------------------------------------------

\begin{rSection}{Academic Achievements}
{\bf Jane Street Graduate Research Fellowship Award Honorable mention (2024)}

{\bf Professor Michael Edelstein Memorial Graduate Prize (2023)}
\\
Annual prize awarded to a graduate student who shows great promise in the mathematical sciences. 


{\bf Honours undergraduate research project (2018)}
\\
Received an award for best undergraduate research project on the algebraic and combinatorial properties of Edge Ideals.
\\
Advisor: Seyed Hamid Hassanzadeh
\end{rSection}

\begin{rSection}{Papers}

\begin{enumerate}
    \item \textit{The weak Lefschetz property of whiskered graphs}.  Lefschetz Properties. SLP-WLP 2022. Springer INdAM Series, vol 59. Springer, Singapore. \href{https://doi.org/10.1007/978-981-97-3886-1_5}{doi.org/10.1007/978-981-97-3886-1\_5}.  With Susan M. Cooper, Sara Faridi, Lisa Nicklasson, Adam Van Tuyl.  \href{https://arxiv.org/abs/2306.04393}{arxiv:2306.04393}
    \item \textit{The weak Lefschetz property and mixed multiplicities of monomial ideals}. J Algebr Comb 60, 295–317 (2024). \href{https://doi.org/10.1007/s10801-024-01337-8}{https://doi.org/10.1007/s10801-024-01337-8}. \href{https://arxiv.org/abs/2306.13274}{arxiv:2306.13274}.
    \item \textit{Spherical complexes}. (2023) With Sara Faridi. (Submitted)% \href{https://arxiv.org/abs/2311.07727}{arXiv:2311.07727}
    \item \textit{Lefschetz properties of squarefree monomial ideals via Rees algebras}. (2024) \href{https://arxiv.org/abs/2404.12471}{arxiv:2404.12471}. (Submitted) 
\end{enumerate}
\end{rSection}

% \begin{rSection}{Preprints at final stage}

%     \begin{enumerate}
%         \item \textit{Realizing resolutions of powers of extremal ideals}. With Trung Chau, Art Duval, Sara Faridi, Susan Morey, Liana \c{S}ega.
%         \item \textit{Pseudo-manifolds arising from very well-covered graphs and grafted simplicial complexes}. With Susan M. Cooper, Sara Faridi, Lisa Nicklasson, Adam Van Tuyl. 
%         \item \textit{Roller coasters, Perazzo forms and independence polynomials}. With Susan M. Cooper, Sara Faridi, Lisa Nicklasson, Adam Van Tuyl.
%         \item \textit{Morse Complex}. With Louis Bu, Sara Faridi, Iresha Madduwe Hewalage, Hasan Mahmood, Dharm Veer, Kyle Wang and Scot Wesley.
%     \end{enumerate}
% \end{rSection}

% \begin{rSection}{Papers in progress}

%     \begin{enumerate}
%         \item \textit{Coinvariant stresses, inverse systems and Lefschetz properties} (Tentative title)
%         \item \textit{Edge ideals of weighted oriented forests via polarization}. With Manohar Kumar.  (Tentative title)
%         \item \textit{Macaulay duality, complete intersections and graphical arrangements}. With Nancy Abdallah. (Tentative title)
%         \item \textit{Probabilistic parking processes}. With Steve Butler, Pamela E. Harris, J. Carlos Martínez Mori, Amanda Priestley, Keith Sullivan, Per Wagenius.
%         \item \textit{Geometric vertex decomposability via binary trees}. With Sergio da Silva. (Tentative title)
%         \item \textit{Planar ternary graphs, flag spheres and Delannoy numbers}. With Margaret Bayer, Rick Danner, Marie Kramer, Yirong Yang. (Tentative title)
%         \item \textit{Algebraic properties of extremal ideals}. With Trung Chau, Art Duval, Sara Faridi, Susan Morey, Liana \c{S}ega. (Tentative title)
%     \end{enumerate}
    
% \end{rSection}
%----------------------------------------------------------------------------------------
%	POSITION OF RESPONSIBILITY SECTION
%----------------------------------------------------------------------------------------

% \newpage
 
\begin{rSection}{Presentations}
    \begin{enumerate}
        \item \textit{When PDEs meet (Combinatorial) Commutative Algebra II}. Dalhousie Commutative Algebra seminar. November 2024.
        \item \textit{When PDEs meet (Combinatorial) Commutative Algebra I}. Dalhousie Commutative Algebra seminar. October 2024.
        \item \textit{Simplicial complexes arising from integer programs}. Graduate Online Combinatorics Colloquium. October 2024. (online invited talk)
        \item \textit{Inverse PDE problems in Geometry}. Dalhousie graduate student seminar. September 2024. (talk)
        \item \textit{Inverse PDE problems in Combinatorics}. Dalhousie honours student seminar. September 2024. (talk)
        \item \textit{Lefschetz properties and analytic spread}. Combinatorics and Geometry in Ioaninna. September 2024 (talk)
        \item \textit{Roller-Coaster, graphs and Perazzo forms}. Uwefest, Notre Dame university. August 2024 (poster)
        \item \textit{Roller-Coaster, graphs and Perazzo forms}. Joint AMS-UMI meeting, Palermo. July 2024 (invited talk)
        \item \textit{Lefschetz properties of squarefree monomial ideals via Rees algebras}. Lefschetz properties in Algebra, Geometry, Topology and Combinatorics, Poland. June 2024 (talk) 
        \item \textit{Lefschetz properties of monomial ideals via Rees algebras}. AMS Sectional meeting, San Francisco. May 2024 (invited talk) 
        \item \textit{Lefschetz properties of monomial ideals via Rees algebras}. COMA/NAG Graduate seminar, \newline MSRI/SLMath. May 2024 (invited talk) 
        \item \textit{Lefschetz properties and mixed multiplicities}. Recent developments in Commutative Algebra, MSRI/SLMath. April 2024 (poster)
        \item \textit{Powers of a simplex: Resolutions meet Partitions}. Combinatorial Algebra meets Algebraic Combinatorics, Montreal. January 2024 (talk)
        \item \textit{Positivity through analytic spread}. Connections workshop, MSRI/SLMath. January 2024 (poster)
        \item \textit{Rees algebras and Lefschetz properties of squarefree monomial ideals}. Commutative Algebra section: Canadian Mathematics Society winter meeting, Montreal. December 2023 (invited talk)
        \item \textit{Rees algebras and Lefschetz properties of squarefree monomial ideals}. Stockholm Commutative Algebra seminar. October 2023 (online talk)
        \item \textit{Lefschetz properties of squarefree monomial ideals}. 10th Heidelberg Laureate Forum, Heidelberg University, Germany. September 2023 (poster, lightning talk)
        \item \textit{Algebraic geometry in the wild}. Dalhousie graduate student seminar. October 2023 (talk)
        \item \textit{Unimodality and log-concavity in Mathematics}. Dalhousie honours student seminar. October 2023 (talk)
        \item \textit{Lefschetz properties of squarefree monomial ideals}. BrianFest, University of Lincoln-Nebraska. August 2023 (poster)
        \item \textit{Homological Invariants of ternary graphs}. 34o Colóquio Brasileiro de Matemática, Instituto de Matemática Pura e Aplicada, Brazil. July 2023 (poster)
        \item \textit{The weak Lefschetz property and mixed multiplicities of monomial ideals}. MSRI/SLMath Summer School: Commutative Algebra and its Interaction with Algebraic Geometry, Notre Dame university. May 2023 (poster)
        \item \textit{Degree of the inverse of a birational monomial map}. MSRI/SLMath Summer School: Commutative Algebra and its Interaction with Algebraic Geometry, Notre Dame university. May 2023 (lightning talk)
        \item \textit{The weak Lefschetz property and mixed multiplicities of monomial ideals}. Workshop on Lefschetz Properties in Algebra, Geometry, Topology and Combinatorics, Fields institute. May 2023 (talk)
        \item \textit{Lefschetz properties and mixed multiplicities via hyperplane arrangements}. International Centre for Theoretical Physics workshop, Italy. May 2023 (poster)
        \item \textit{The weak Lefschetz property and mixed multiplicities of squarefree monomial ideals}. International Centre for Theoretical Physics summer school, Italy. May 2023 (talk)
        \item \textit{Homological invariants of ternary graphs}. Banff International Research Station workshop: Interactions Between Topological Combinatorics and Combinatorial Commutative Algebra. April 2023 (talk)
        \item \textit{Homological invariants of ternary graphs}. Southern Regional Algebra Conference, Tulane university. March 2023 (online talk)
        \item \textit{Homological invariants of ternary graphs}. Combinatorial Algebra meets Algebraic Combinatorics, Waterloo University. January 2023 (poster)
        \item \textit{The log-concavity of chromatic polynomials}. Universidade Federal do Rio de Janeiro, Brazil. July 2022 (talk, in portuguese)
        \item \textit{Exploring examples with Macaulay2}. Universidade Federal do Rio de Janeiro, Brazil. July 2022 (talk, in portuguese)
        \item \textit{Hilbert functions in Combinatorics}. Universidade Federal do Rio de Janeiro, Brazil. July 2022 (talk, in portuguese)
        \item \textit{A combinatorial interpretation of the primary decomposition of edge ideals}. Universidade Federal do Rio de Janeiro undergraduate student seminar, Brazil. 2019 (talk, in portuguese)
        \item \textit{A brief dictionary between Algebra and Combinatorics}. Semana de Integração Acadêmica da Universidade Federal do Rio de Janeiro, Brazil 2018 (talk, in portuguese)
    \end{enumerate}
\end{rSection}

% \newpage

\begin{rSection}{Teaching positions}
    {\bf Instructor: Engineering Mathematics II at Dalhousie Univeristy}
    \\
    Summer of 2023
    \\
    {\bf Teaching assistant at Dalhousie university}
    \\
    Fall of 2022 - current
    \\
    {\bf Teaching assistant at Universidade Federal do Rio de Janeiro} 
    \\
    First semester of 2019
    \\
    {\bf Teaching assistant at Universidade Federal do Rio de Janeiro} 
    \\
    First and second semester of 2018
    \\
        
\end{rSection}

\begin{rSection}{Leadership positions}
{\bf Organizer of Week of Applied Mathematics and Mathematical Engineering (SEM$^2$Ap)}
\\
Organized virtual Week of Applied Mathematics and Mathematical Engineering (SEM$^2$Ap) in 2020
{\scriptsize More info on:  \href{http://semap.rio.br/2020/en/}{Webpage},   \href{https://www.youtube.com/channel/UC14NMQ5cOsSuLrAQGGa2T4Q}{Youtube}, 
\href{https://www.instagram.com/semap.rio/}{Instagram}}
\\
{\bf Organizer of weekly student seminar at Universidade Federal do Rio de Janeiro}
\\
Organized weekly seminars at Universidade Federal do Rio de Janeiro during the second semester of 2019.
\end{rSection}

% \clearpage

%----------------------------------------------------------------------------------------
%	EVENTS ATTENDED
%----------------------------------------------------------------------------------------

% \begin{rSection}{Events attended}

%     \begin{itemize}[label={}]
%         \item Combinatorial Algebra meets Algebraic Combinatorics (Montreal) January, 2024
%         \item Introductory Workshop: Commutative Algebra (MSRI/SLMath) January, 2024
%         \item Connections Workshop: Commutative Algebra (MSRI/SLMath) January, 2024
%         \item Canadian Mathematics Society winter meeting (Montreal) December, 2023
%         \item 10th Heidelberg Laureate Forum 2023 (Heidelberg University, Germany) September, 2023
%         \item BrianFest (University of Nebraska-Lincoln) August, 2023
%         \item 34o Colóquio Brasileiro de Matemática (Instituto de Matemática Pura e Aplicada, Brazil) July, 2023
%         \item Commutative Algebra and its Interaction with Algebraic Geometry (University of Notre Dame) May, 2023
%         \item Workshop on Lefschetz Properties in Algebra, Geometry, Topology and Combinatorics. (Fields Institute) May, 2023
%         \item Workshop on Commutative Algebra and Algebraic Geometry in Prime Characteristics. (International Centre for Theoretical Physics, Italy) May, 2023
%         \item School on Commutative Algebra and Algebraic Geometry in Prime Characteristics. (International Centre for Theoretical Physics, Italy) May, 2023
%         \item Interactions Between Topological Combinatorics and Combinatorial Commutative Algebra (Banff International Research Station) April, 2023
%         \item Combinatorial Algebra meets Algebraic Combinatorics 2023. (Waterloo University) January, 2023
%         \item Combinatorial Algebra meets Algebraic Combinatorics 2020. (Dalhousie University) January, 2020
%         \item 1st Joint Meeting Brazil-France in Mathematics. (Instituto de Matemática Pura e Aplicada, Brazil) 2019
%         \item 32o Colóquio Brasileiro de Matemática. (Instituto de Matemática Pura e Aplicada, Brazil) 2019
%         \item SIMCARA. (Universidade São Paulo: São Carlos, Brazil) 2019
%         \item Syzygies, from Theory to Applications. (Universidade Federal da Paraíba, Brazil) 2019
%         \item XXV Escola de Álgebra. (Universidade Estadual de Campinas, Brazil) 2018
%     \end{itemize}
% % \\ 34o Colóquio Brasileiro de Matemática (IMPA) 2023 (poster)
% % \\ Commutative Algebra and its Interaction with Algebraic Geometry (University of Notre Dame) 2023 (poster, lightning talk)
% % \\ Workshop on Lefschetz Properties in Algebra, Geometry, Topology and Combinatorics. (Fields Institute) 2023 (talk, problem)
% % \\ Workshop on Commutative Algebra and Algebraic Geometry in Prime Characteristics. (ICTP) 2023 (poster)
% % \\ School on Commutative Algebra and Algebraic Geometry in Prime Characteristics. (ICTP) 2023 (talk)
% % \\ Interactions Between Topological Combinatorics and Combinatorial Commutative Algebra (BIRS) 2023 (talk)
% % % \\ Southern Regional Algebra Conference 2023. (online talk)
% % \\ Combinatorial Algebra meets Algebraic Combinatorics 2023. (Waterloo University) 2023 (poster)%(\href{http://semap.rio.br}{poster} presented)
% % \\ Combinatorial Algebra meets Algebraic Combinatorics 2020. (Dalhousie University) 2020
% % \\ 1st Joint Meeting Brazil-France in Mathematics. (IMPA) 2019
% % \\ 32o Colóquio Brasileiro de Matemática. (IMPA) 2019
% % \\ SIMCARA. (USP São Carlos) 2019
% % \\ Syzygies, from Theory to Applications. (UFPB) 2019
% % \\ XXV Escola de Álgebra. (UNICAMP) 2018
% % \\
% \end{rSection}

% \newpage

\begin{rSection}{Skills}

    \begin{tabular}{ @{} >{\bfseries}l @{\hspace{6ex}} l }
    Programming:\ & Macaulay2, Sage, Python, Javascript, Julia
    \\
    \end{tabular}
    
    \end{rSection}

% \begin{rSection}{Research line}
% I am interested in the study of Combinatorial Commutative Algebra, Computational Algebra and their applications.

% \end{rSection}


\end{document}
